\documentclass[a4paper]{article}

\usepackage{ifxetex}

\ifxetex
    \usepackage{fontspec}
    \setmainfont{PT Serif}
\fi

\usepackage[margin=5em]{geometry}
\usepackage{parskip}

\begin{document}

\begin{center}
{\huge \textbf{Extreme Computing Notes}}
\end{center}

\section{Computing as a Service}

\subsection*{Patter Buster}

\begin{itemize}
\item
  \textbf{Infrastructure as a Service} \textit{(Utility Computing)}
  \begin{itemize}
  \item
    The vendor provides and maintains the hardware, everything else is up to the client.
  \item
    \textit{Amazon EC2, Rackspace}
  \end{itemize}
\item
  \textbf{Platform as a Service}
  \begin{itemize}
  \item
    The hardware and underlying operating system are abstracted away from the client, providing a platform for the development of and deployment of applications
  \item
    \textit{Heroku, Google App Engine}
  \end{itemize}
\item
  \textbf{Software as a Service}
  \begin{itemize}
  \item
    Where the software is delivered to the client from a centrally hosted location. Mostly intended for end-users.
    \item
  \textit{GMail, Salesforce}

  \end{itemize}
\end{itemize}

\subsection*{Why?}

\begin{itemize}
\item
  Addresses issues of cost and feasability in regard to scalability
\item
  Provides \textit{elasticity}, such that resources required can be scaled in accordance with demand
\end{itemize}

\section{The Cloud}

The \textit{first tier} handles client requests, should be lightweight and can be handled by simple PHP pages for example. \textit{Replication} is akey concept such that the tier can handled most client requests without delay. This tier also takes away strain from the `online load' from the second tier.

The \textit{second tier} performs much of the heavy lifting and makes extensive use of \textit{caching}, but does not need replication to the same extent of the first tier.



\end{document}

