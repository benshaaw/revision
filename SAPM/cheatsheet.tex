\documentclass[10pt,landscape]{article}
\usepackage{multicol}
\usepackage{calc}
\usepackage{ifthen}
\usepackage[landscape]{geometry}
\usepackage{amsmath,amsthm,amsfonts,amssymb}
\usepackage{color,graphicx,overpic}
\usepackage{hyperref}


\pdfinfo{
  /Title (example.pdf)
  /Creator (TeX)
  /Producer (pdfTeX 1.40.0)
  /Author (Seamus)
  /Subject (Example)
  /Keywords (pdflatex, latex,pdftex,tex)}

% This sets page margins to .5 inch if using letter paper, and to 1cm
% if using A4 paper. (This probably isn't strictly necessary.)
% If using another size paper, use default 1cm margins.
\ifthenelse{\lengthtest { \paperwidth = 11in}}
    { \geometry{top=.5in,left=.5in,right=.5in,bottom=.5in} }
    {\ifthenelse{ \lengthtest{ \paperwidth = 297mm}}
        {\geometry{top=1cm,left=1cm,right=1cm,bottom=1cm} }
        {\geometry{top=1cm,left=1cm,right=1cm,bottom=1cm} }
    }

% Turn off header and footer
\pagestyle{empty}

% Redefine section commands to use less space
\makeatletter
\renewcommand{\section}{\@startsection{section}{1}{0mm}%
                                {-1ex plus -.5ex minus -.2ex}%
                                {0.5ex plus .2ex}%x
                                {\normalfont\large\bfseries}}
\renewcommand{\subsection}{\@startsection{subsection}{2}{0mm}%
                                {-1explus -.5ex minus -.2ex}%
                                {0.5ex plus .2ex}%
                                {\normalfont\normalsize\bfseries}}
\renewcommand{\subsubsection}{\@startsection{subsubsection}{3}{0mm}%
                                {-1ex plus -.5ex minus -.2ex}%
                                {1ex plus .2ex}%
                                {\normalfont\small\bfseries}}
\makeatother

% Define BibTeX command
\def\BibTeX{{\rm B\kern-.05em{\sc i\kern-.025em b}\kern-.08em
    T\kern-.1667em\lower.7ex\hbox{E}\kern-.125emX}}

% Don't print section numbers
\setcounter{secnumdepth}{0}


\setlength{\parindent}{0pt}
\setlength{\parskip}{0pt plus 0.5ex}

%My Environments
\newtheorem{example}[section]{Example}
% -----------------------------------------------------------------------

\begin{document}
\raggedright
\footnotesize
\begin{multicols}{2}


% multicol parameters
% These lengths are set only within the two main columns
%\setlength{\columnseprule}{0.25pt}
\setlength{\premulticols}{1pt}
\setlength{\postmulticols}{1pt}
\setlength{\multicolsep}{1pt}
\setlength{\columnsep}{2pt}

\section{Common Quality Attributes}
\begin{itemize}
\item Availability - Need to be there when required i.e can always get a call through
\item Performance - Control resource demand tactics i.e robot voice doesn't start mid call
\item Modifiability - Can change an update system i.e want to add video calls
\item Reliability - System does not fail i.e call does not drop
\item Security - System is not vunerable i.e. people can't listen in
\item Testability - Possible to test in any way required
\end{itemize}

\section{Common Stakeholders}
\begin{itemize}
\item User
\item Service Provider
\item Analysis
\item Developer
\end{itemize}

\section{Common Design Patterns}
\begin{itemize}
\item Layered - distict layers, can only interect with adjacent layers
\item Model-View-Controller (MVC) - UI and application isolation, user intarface changes often
\item Blackboard - think /r/place
\item N-tier Architecture - See Layered
\end{itemize}

\section{Types of Structures}
\begin{itemize}
\item Modular - static structures, focus on how functionality is divided up
\item Component \& Connector - Runtime tructures focused on interactions
\item Allocation - Mapping to Environments
\end{itemize}

\section{Lifecycles}
\begin{itemize}
\item V Model
  \begin{itemize}
  \item ``Verification \& Validation'' model
  \item Extension of waterfall model
  \item Testing planned in parallel with development
  \item Coding phase joins these two sides
  \end{itemize}
\item RUP
  \begin{itemize}
  \item ``Rational Unified Process''
    \item Kind of like a gantt chart
   \end{itemize}
\item Spiral Model
  \begin{itemize}
  \item Risk driven process model
  \item Itterative design model
  \end{itemize}
\item Agile SCRUM
  \begin{itemize}
  \item 2-4 week sprint
  \item Daily scrum meetings
  \item Most important issues tackled first as they are on top of backlog
  \end{itemize}
\end{itemize}
%\section{QA Testing Senarios}
%\begin{itemize}
%\item Availability
%\item Performance
%\item Modifiability
%\item Reliability
%\item Security
%\item Testability
%\end{itemize}
% You can even have references
\end{multicols}
\end{document}
