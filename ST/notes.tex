\documentclass{article}
\usepackage[margin=2.5cm]{geometry}
\usepackage{parskip}
\usepackage{hyperref}

\usepackage{fontspec}
\usepackage{xltxtra}

% PT Serif
\setromanfont[ BoldFont=PTF75F.ttf, ItalicFont=PTZ56F.ttf,
BoldItalicFont=PTF76F.ttf, ]{PTF55F.ttf}

% Noto Sans
\setsansfont[ BoldFont=NotoSans-Bold.ttf,
ItalicFont=NotoSans-Italic.ttf, BoldItalicFont=NotoSans-BoldItalic.ttf
]{NotoSans-Regular.ttf}

\hypersetup{ colorlinks = false }


\begin{document}
\pagestyle{headings}
\textbf{\huge Software Testing Notes}\\
\textit{\footnotesize Found at: \href{http://benjaminshaw.uk}{benjaminshaw.uk}}

\section{Terms}

\textbf{Black-Box Testing}

Examine the functionality of an application without looking at its inner workings.

\textbf{White-Box Testing}

Testing the internal structures of an application

\section{Functional Testing}

A type of testing that focuses on \textit{what} the system does. Utilising \textit{black-box testing}., test cases are derived on the specification of the component under scrutiny.

Functional testing does not look at methods of a module, rather it tests a \textit{slice of the whole program's functionality.}

\section{Combinatorial Testing}

\section{Structural Testing}

\section{Dependence and Data Flow Models}

\section{Data Flow Testing}

\section{Model Based Testing}

\section{Testing OOP Software}

\section{Mutation Testing}

\section{Regression Testing}

\section{Integration and Component-Based Testing}

\end{document}